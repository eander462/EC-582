% Options for packages loaded elsewhere
\PassOptionsToPackage{unicode}{hyperref}
\PassOptionsToPackage{hyphens}{url}
%
\documentclass[
]{article}
\usepackage{amsmath,amssymb}
\usepackage{lmodern}
\usepackage{iftex}
\ifPDFTeX
  \usepackage[T1]{fontenc}
  \usepackage[utf8]{inputenc}
  \usepackage{textcomp} % provide euro and other symbols
\else % if luatex or xetex
  \usepackage{unicode-math}
  \defaultfontfeatures{Scale=MatchLowercase}
  \defaultfontfeatures[\rmfamily]{Ligatures=TeX,Scale=1}
\fi
% Use upquote if available, for straight quotes in verbatim environments
\IfFileExists{upquote.sty}{\usepackage{upquote}}{}
\IfFileExists{microtype.sty}{% use microtype if available
  \usepackage[]{microtype}
  \UseMicrotypeSet[protrusion]{basicmath} % disable protrusion for tt fonts
}{}
\makeatletter
\@ifundefined{KOMAClassName}{% if non-KOMA class
  \IfFileExists{parskip.sty}{%
    \usepackage{parskip}
  }{% else
    \setlength{\parindent}{0pt}
    \setlength{\parskip}{6pt plus 2pt minus 1pt}}
}{% if KOMA class
  \KOMAoptions{parskip=half}}
\makeatother
\usepackage{xcolor}
\IfFileExists{xurl.sty}{\usepackage{xurl}}{} % add URL line breaks if available
\IfFileExists{bookmark.sty}{\usepackage{bookmark}}{\usepackage{hyperref}}
\hypersetup{
  pdftitle={HW1},
  hidelinks,
  pdfcreator={LaTeX via pandoc}}
\urlstyle{same} % disable monospaced font for URLs
\usepackage[margin=1in]{geometry}
\usepackage{color}
\usepackage{fancyvrb}
\newcommand{\VerbBar}{|}
\newcommand{\VERB}{\Verb[commandchars=\\\{\}]}
\DefineVerbatimEnvironment{Highlighting}{Verbatim}{commandchars=\\\{\}}
% Add ',fontsize=\small' for more characters per line
\usepackage{framed}
\definecolor{shadecolor}{RGB}{248,248,248}
\newenvironment{Shaded}{\begin{snugshade}}{\end{snugshade}}
\newcommand{\AlertTok}[1]{\textcolor[rgb]{0.94,0.16,0.16}{#1}}
\newcommand{\AnnotationTok}[1]{\textcolor[rgb]{0.56,0.35,0.01}{\textbf{\textit{#1}}}}
\newcommand{\AttributeTok}[1]{\textcolor[rgb]{0.77,0.63,0.00}{#1}}
\newcommand{\BaseNTok}[1]{\textcolor[rgb]{0.00,0.00,0.81}{#1}}
\newcommand{\BuiltInTok}[1]{#1}
\newcommand{\CharTok}[1]{\textcolor[rgb]{0.31,0.60,0.02}{#1}}
\newcommand{\CommentTok}[1]{\textcolor[rgb]{0.56,0.35,0.01}{\textit{#1}}}
\newcommand{\CommentVarTok}[1]{\textcolor[rgb]{0.56,0.35,0.01}{\textbf{\textit{#1}}}}
\newcommand{\ConstantTok}[1]{\textcolor[rgb]{0.00,0.00,0.00}{#1}}
\newcommand{\ControlFlowTok}[1]{\textcolor[rgb]{0.13,0.29,0.53}{\textbf{#1}}}
\newcommand{\DataTypeTok}[1]{\textcolor[rgb]{0.13,0.29,0.53}{#1}}
\newcommand{\DecValTok}[1]{\textcolor[rgb]{0.00,0.00,0.81}{#1}}
\newcommand{\DocumentationTok}[1]{\textcolor[rgb]{0.56,0.35,0.01}{\textbf{\textit{#1}}}}
\newcommand{\ErrorTok}[1]{\textcolor[rgb]{0.64,0.00,0.00}{\textbf{#1}}}
\newcommand{\ExtensionTok}[1]{#1}
\newcommand{\FloatTok}[1]{\textcolor[rgb]{0.00,0.00,0.81}{#1}}
\newcommand{\FunctionTok}[1]{\textcolor[rgb]{0.00,0.00,0.00}{#1}}
\newcommand{\ImportTok}[1]{#1}
\newcommand{\InformationTok}[1]{\textcolor[rgb]{0.56,0.35,0.01}{\textbf{\textit{#1}}}}
\newcommand{\KeywordTok}[1]{\textcolor[rgb]{0.13,0.29,0.53}{\textbf{#1}}}
\newcommand{\NormalTok}[1]{#1}
\newcommand{\OperatorTok}[1]{\textcolor[rgb]{0.81,0.36,0.00}{\textbf{#1}}}
\newcommand{\OtherTok}[1]{\textcolor[rgb]{0.56,0.35,0.01}{#1}}
\newcommand{\PreprocessorTok}[1]{\textcolor[rgb]{0.56,0.35,0.01}{\textit{#1}}}
\newcommand{\RegionMarkerTok}[1]{#1}
\newcommand{\SpecialCharTok}[1]{\textcolor[rgb]{0.00,0.00,0.00}{#1}}
\newcommand{\SpecialStringTok}[1]{\textcolor[rgb]{0.31,0.60,0.02}{#1}}
\newcommand{\StringTok}[1]{\textcolor[rgb]{0.31,0.60,0.02}{#1}}
\newcommand{\VariableTok}[1]{\textcolor[rgb]{0.00,0.00,0.00}{#1}}
\newcommand{\VerbatimStringTok}[1]{\textcolor[rgb]{0.31,0.60,0.02}{#1}}
\newcommand{\WarningTok}[1]{\textcolor[rgb]{0.56,0.35,0.01}{\textbf{\textit{#1}}}}
\usepackage{graphicx}
\makeatletter
\def\maxwidth{\ifdim\Gin@nat@width>\linewidth\linewidth\else\Gin@nat@width\fi}
\def\maxheight{\ifdim\Gin@nat@height>\textheight\textheight\else\Gin@nat@height\fi}
\makeatother
% Scale images if necessary, so that they will not overflow the page
% margins by default, and it is still possible to overwrite the defaults
% using explicit options in \includegraphics[width, height, ...]{}
\setkeys{Gin}{width=\maxwidth,height=\maxheight,keepaspectratio}
% Set default figure placement to htbp
\makeatletter
\def\fps@figure{htbp}
\makeatother
\setlength{\emergencystretch}{3em} % prevent overfull lines
\providecommand{\tightlist}{%
  \setlength{\itemsep}{0pt}\setlength{\parskip}{0pt}}
\setcounter{secnumdepth}{-\maxdimen} % remove section numbering
\ifLuaTeX
  \usepackage{selnolig}  % disable illegal ligatures
\fi

\title{HW1}
\author{}
\date{\vspace{-2.5em}2023-03-30}

\begin{document}
\maketitle

\begin{Shaded}
\begin{Highlighting}[]
\CommentTok{\# Load packages}
\NormalTok{pacman}\SpecialCharTok{::}\FunctionTok{p\_load}\NormalTok{(tidyverse, magrittr)}
\end{Highlighting}
\end{Shaded}

\hypertarget{question-1}{%
\subsubsection{Question 1}\label{question-1}}

\begin{Shaded}
\begin{Highlighting}[]
\CommentTok{\# Set true values of parameters}
\NormalTok{y }\OtherTok{=} \FunctionTok{c}\NormalTok{(}\DecValTok{1}\NormalTok{, }\DecValTok{1}\NormalTok{)}
\NormalTok{sigma }\OtherTok{=} \FunctionTok{matrix}\NormalTok{(}\FunctionTok{diag}\NormalTok{(}\FunctionTok{c}\NormalTok{(}\FloatTok{0.25}\SpecialCharTok{\^{}}\DecValTok{2}\NormalTok{,}\DecValTok{1}\NormalTok{)), }\AttributeTok{ncol =} \DecValTok{2}\NormalTok{)}
\NormalTok{theta }\OtherTok{=}\NormalTok{ y[}\DecValTok{1}\NormalTok{]}\SpecialCharTok{/}\NormalTok{y[}\DecValTok{2}\NormalTok{]}

\CommentTok{\# Simulating parameters}
\NormalTok{n }\OtherTok{=} \DecValTok{50}
\NormalTok{B }\OtherTok{=} \DecValTok{10000}

\CommentTok{\# Set seed}
\FunctionTok{set.seed}\NormalTok{(}\DecValTok{123}\NormalTok{)}

\CommentTok{\# Initialize simulation vectors}
\NormalTok{sim.theta }\OtherTok{=} \FunctionTok{rep}\NormalTok{(}\DecValTok{0}\NormalTok{, B)}
\NormalTok{sim.mu1 }\OtherTok{=} \FunctionTok{rep}\NormalTok{(}\DecValTok{0}\NormalTok{,B)}
\NormalTok{sim.mu2 }\OtherTok{=} \FunctionTok{rep}\NormalTok{(}\DecValTok{0}\NormalTok{, B)}

\CommentTok{\# Monte Carlo loop}
\ControlFlowTok{for}\NormalTok{ (sim }\ControlFlowTok{in} \DecValTok{1}\SpecialCharTok{:}\NormalTok{B) \{}
  \CommentTok{\# Simulate data}
\NormalTok{  sim.y }\OtherTok{=} \FunctionTok{tibble}\NormalTok{(}\AttributeTok{y1 =} \FunctionTok{rnorm}\NormalTok{(n, }\AttributeTok{mean =}\NormalTok{ y[}\DecValTok{1}\NormalTok{], }\AttributeTok{sd =}\NormalTok{ sigma[}\DecValTok{1}\NormalTok{,}\DecValTok{1}\NormalTok{]), }\AttributeTok{y2 =} \FunctionTok{rnorm}\NormalTok{(n, }\AttributeTok{mean =}\NormalTok{ y[}\DecValTok{2}\NormalTok{], }\AttributeTok{sd =}\NormalTok{ sigma[}\DecValTok{2}\NormalTok{,}\DecValTok{2}\NormalTok{]))}
\NormalTok{  mu\_hat }\OtherTok{=} \FunctionTok{c}\NormalTok{(}\FunctionTok{mean}\NormalTok{(sim.y}\SpecialCharTok{$}\NormalTok{y1), }\FunctionTok{mean}\NormalTok{(sim.y}\SpecialCharTok{$}\NormalTok{y2))}
\NormalTok{  sigma\_hat }\OtherTok{=} \FunctionTok{with}\NormalTok{(sim.y, }\FunctionTok{c}\NormalTok{(}\FunctionTok{sd}\NormalTok{(y1), }\FunctionTok{rep}\NormalTok{(}\FunctionTok{cov}\NormalTok{(y1, y2), }\DecValTok{2}\NormalTok{), }\FunctionTok{sd}\NormalTok{(y2))) }\SpecialCharTok{|}\ErrorTok{\textgreater{}} \FunctionTok{matrix}\NormalTok{(}\AttributeTok{ncol =} \DecValTok{2}\NormalTok{)}
  
  \CommentTok{\# Parameter of interest}
\NormalTok{  theta\_hat }\OtherTok{=}\NormalTok{ mu\_hat[}\DecValTok{1}\NormalTok{]}\SpecialCharTok{/}\NormalTok{mu\_hat[}\DecValTok{2}\NormalTok{]}
  
  \CommentTok{\# Store the simulated values }
\NormalTok{  sim.theta[sim] }\OtherTok{=}\NormalTok{ theta\_hat}
\NormalTok{  sim.mu1[sim] }\OtherTok{=}\NormalTok{ mu\_hat[}\DecValTok{1}\NormalTok{]}
\NormalTok{  sim.mu2[sim] }\OtherTok{=}\NormalTok{ mu\_hat[}\DecValTok{2}\NormalTok{]}
\NormalTok{\}}

\CommentTok{\# Calculate the mean and standard error of theta\_hat}
\NormalTok{mean\_theta\_hat }\OtherTok{=} \FunctionTok{mean}\NormalTok{(sim.theta); mean\_theta\_hat}
\end{Highlighting}
\end{Shaded}

\begin{verbatim}
## [1] 1.020602
\end{verbatim}

\begin{Shaded}
\begin{Highlighting}[]
\NormalTok{se\_theta\_hat }\OtherTok{=} \FunctionTok{sd}\NormalTok{(sim.theta); se\_theta\_hat}
\end{Highlighting}
\end{Shaded}

\begin{verbatim}
## [1] 0.1531086
\end{verbatim}

\begin{Shaded}
\begin{Highlighting}[]
\CommentTok{\# Confidence interval of theta\_hat}
\NormalTok{mean\_theta\_hat }\SpecialCharTok{+} \FunctionTok{c}\NormalTok{(}\SpecialCharTok{{-}}\NormalTok{se\_theta\_hat, se\_theta\_hat)}\SpecialCharTok{*}\FloatTok{1.96}
\end{Highlighting}
\end{Shaded}

\begin{verbatim}
## [1] 0.7205087 1.3206943
\end{verbatim}

\hypertarget{question-2}{%
\subsubsection{Question 2}\label{question-2}}

Theta is defined as \(\theta = \frac{\mu_1}{\mu_2} = f(\mu)\). To use
the delta method, we take the derivative of \(\theta\) in therms of
\(\mu_1 \text{ and } \mu_2\).

\[
g(\theta) = \frac{\partial f(\mu)}{\partial\mu} = \left(\begin{array}{c}
\frac{\partial f(\mu)}{\partial \mu_1}  \\
\frac{\partial f(\mu)}{\partial \mu_2}  \\
\end{array}
\right) = \left(\begin{array}{c})
\frac{1}{\mu_2}
\frac{-\mu_1}{\mu_2^2}
\right)
\]

\end{document}
